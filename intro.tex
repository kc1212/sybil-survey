Reputation systems (described in \autoref{sec:reputation}) allow entites,
usually humans, to trust each other in the cyberspace based on their prior
interactions (logical) or knowledge from other entities. For instance, online
marketplaces such as Aamazon or eBay often use a reputation system, causing new
buyers to have a higher likelihood to buy goods from merchents with a high
rating (a metric for reputation) because a lot of other buyers left a positive
feedback.

However, reputation systems are vulnerable to many types of attacks. The
sybil-attack, first described by Douceur\cite{douceur2002sybil}, is an attack
where an entity can assume multiple identities or sybils, and then attack either
another entity or undermine the whole reputation system (we discuss it in more
details in \autoref{sec:sybil}). In the marketplace example, the merchent could
create multiple fake accounts and submitting a lot of positive feedback to the
real account to boost the rating. It is one of the most important attacks
because it leads to a large number of consequences including but not limited to
spreading false information, ballot stuffing\cite{bhattacharjee2005avoiding} and
eclipse attacks\cite{singh2006eclipse}. Thus, preventing the sybil-attack is
likely to significantly increase the credibility of reputaiton systems.

Sybil-defence mechanisms come in various shapes and sizes. Some rely on a
trusted third party (\autoref{sec:trusted_party}), some introduce a cost in
identity creation (\autoref{sec:costly_id}), some exploit the graph
characteristics (\autoref{sec:graph}) and so on. To the best our knowledge,
there does not exist a recent and comprehensive survey that focuses on the
sybil-attack in reputation systems.

% scope of this work
In this work, we survey the defence mechanisms proposed by various reputation
systems to eliminate or minimise sybil-attacks as well as general approaches
that do not depend on any specific reputation system. Sybil-attacks do not only
exist in reputation systems. Wireless sensor networks for example are also
vulnerable, the attacker can cripple the routing algorithm or defeat distributed
storage mechanisms\cite{newsome2004sybil}. However, wireless sensor networks do
not usually involve reputation systems, thus it is not covered.

Our main contributions are the following.
\begin{enumerate}
  \item TODO
  \item TODO
\end{enumerate}

%%% Local Variables:
%%% mode: latex
%%% TeX-master: "main"
%%% End:
