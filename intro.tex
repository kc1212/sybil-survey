Electronic commerce and online social networking are common events at the
present time. They allow us to orchestrate many aspects of our lives in the
comfort of our homes, behind the monitors of our devices. An online identity is
often required to use such services, for examples we must create an account to
use tweet\footnote{A message sent using Twitter is a tweet.} our friend who must
also have an account. In this scenario, users can choose to remain psudonymous
if they are careful, where their real-life identity is uncorrelated with their
online identity. % privacy

While creating pseudonyms is a useful for protecting users' privacy, it also
opens a alleyway for attackers. The Sybil attack, first described by
Douceur\cite{douceur2002sybil}, is an attack where an entity can assume multiple
identities or Sybils, and then attack either another entity or undermine the
whole system. For example, a malicious twitter user can create many fake
identities and have the fake identities follow his read identity, thus creating
a false reputation. It is one of the most important attacks because it leads to
a large number of consequences including but not limited to spreading false
information, ballot stuffing\cite{bhattacharjee2005avoiding} and eclipse
attacks\cite{singh2006eclipse}. Furthermore, to the best of our knowledge, there
is no general solution for preventing the Sybil attack.

In this work, we survey both the theory and practical aspects of the
Sybil attack. Section \ref{sec:sybil} describes the Sybil attack in more detail
and outlines some of the theoretical results. Then we look at how researchers
and black-hat hackers mounted the Sybil attack on real-world e-commerce and
online social network systems. There is a large variety of Sybil attack defence
mechanisms, from using trusted-third-party to exploiting the graph
characteristics in online social networks, thus we classify those mechanisms
by their ``main idea'' in \ref{sec:defences}. Finally we present the related
work in this area in Section \ref{sec:related}.

%%% Local Variables:
%%% mode: latex
%%% TeX-master: "main"
%%% End:
