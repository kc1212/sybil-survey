% TODO expand, better intro
In this section we categorise various defence techniques against the
sybil-attack in reputation systems.

\subsection{Trusted Third Party}\label{sec:trusted_party}
One of the earliest and best known reputation system is
eBay\cite{resnick2002trust}. The buyers and sellers rely on a trusted third
party, in this case eBay, to gather and distribute feedbacks after every
transaction. Even when there are no incentives to provide feedback, Resnick and
Zeckhauser observed that feedback was provided more than half of the
time\cite{resnick2002trust}, making eBay one of the most well-known online
marketplaces.

In general, trusted third parties manage the issurance and verification of
identities. Thus they can apply a fee on the peer for creating a new
identity\cite{resnick2001social} or rate-limit the creation of new
identities\cite{douceur2002sybil}, making sybil-attacks more difficult.
Furthermore, trusted third parties often have the ability to manipulate the
identities. For example they could punish the attackers by disabling all of
their identity when caught, making the sybil-attack much riskier especially when
identities are costly.

Trusted third party is likely the most widely used technique in practice.
Marketplaces such as Amazon or eBay, online forums such as Stackoverflow or
Reddit, all use a form of trusted third party.

Unfortunately, a trusted third party is often a single point of failure.
Moreover, being a centralised system, it is difficult to scale up to suit
increasing user demands. % TODO GIVE exmaples of failures
In the remainder of this section, we focus on distributed techniques for
preventing the sybil-attack.

\subsection{Costly Identity Creation}\label{sec:costly_id}
\subsubsection{IP Address}
\subsubsection{Low reputation for new users}

\subsection{Graph Techniques}\label{sec:graph}
BarterCast\cite{meulpolder2009bartercast}
EigenTrust\cite{kamvar2003eigentrust}
Social-network\cite{viswanath2010analysis}
SybilGuard\cite{yu2006sybilguard}
SybilLimit\cite{yu2008sybillimit}
Theory\cite{seuken2011sybil}

\subsection{Reputation Transfer}
Trust-transfer\cite{seigneur2005trust}

\subsection{Blockchain Based Techniques?}
Privacy-preserving\cite{schaub2016trustless}
Proof-of-stake\cite{dennis2016rep}

\subsection{Unsorted?}
SybilInfer\cite{danezis2009sybilinfer}
Sybil-proof\cite{cheng2005sybilproof}
Self-registration\cite{dinger2006defending}
Secure-Overlay\cite{lua2007securing}
SybilProof-DHT\cite{lesniewski2010whanau}
TrustMe\cite{singh2003trustme} is a reputation that focuses on anonymity.

Beth 94\cite{beth1994valuation}
PGP (Zimmermann) 95\cite{zimmermann1995official}
Yu 00\cite{yu2000social}
P-GRID 01\cite{aberer2001p}
CORE 02\cite{michiardi2002core}
XRep 02\cite{damiani2002reputation}
Lee 03\cite{lee2003cooperative}
Xiong 03\cite{xiong2003reputation}
% Buchegger 04 - MANETs
Feldman 04\cite{feldman2004robust}
Guha 04\cite{guha2004propagation}
Marti 04\cite{marti2004limited}
ARA 05\cite{ham2005ara}
Scrivener 05\cite{nandi2005scrivener}
Song 05\cite{song2005trusted}
TrustGuard 05\cite{srivatsa2005trustguard}
Xiong 05\cite{xiong2007countering}
PowerTrust 06\cite{zhou2007powertrust}
Credence 06\cite{walsh2006experience}
P2PRep/Fuzzy 06\cite{aringhieri2006fuzzy}
% Li 07 - MANETs

Histos
Sopras
Regret
Beta
Confidant
Gupta et al.
PeerTrust
Ismail et al.
Pride
FuzzyTrust
Travos
Gal-Oz et al.
Coner et al.
H-Trust
RateWeb
Tong and Zhang
R2Trust
ReDS
Tulungan
GRAft
PerContRep


%%% Local Variables:
%%% mode: latex
%%% TeX-master: "main"
%%% End:
