The Sybil attack is coined by Douceur\cite{douceur2002sybil} in 2002 in the
context of peer-to-peer systems. In this section, we first introduce the Sybil
attack using Douceur's original definition and outline the key (discouraging)
theoretical results. Next, we review practical attacks in three types of systems
(1) MANETs (mobile ad-hoc networks) such as sensor networks, (2) reputation
systems such as eBay or PageRank\cite{page1999pagerank} and (3) OSN (online
social networks) such as Twitter and Facebook. We hope our review illuminates
the alarming consequences of the Sybil attack.

\subsection{Theoretical Results}
Douceur defined the Sybil attack as forging multiple identities under the same
entity\cite{douceur2002sybil}. An entity can be for example a physical user of
the system and identities are how entities present themselves to the system.
Thus a local entity has no direct knowledge of remote entities, only their
identities. We use these terms in the remainder of the survey. The author
modelled the system as a general distributed computing environment where there
is no constraint on the topology, every node has limited computational resources
and messages are guaranteed to be delivered. Under this model, the author proved
that the Sybil attack is always possible without a central, trusted authority.

% Preventing the Sybil attack is in fact a lot
% more difficult because peer-to-peer systems often do not have a central, trusted
% authority.

Cheng and Friedman proved an important result regarding the Sybil attack in
reputation systems\cite{cheng2005sybilproof}. Reputation systems are commonly
used in MANETs, e-commerce and the internet in general, where entities are
rewarded by their good behaviour and penalised otherwise. Google's
PageRank\cite{page1999pagerank} is an example of a reputation system, where a
large number of links to a website makes it more reputable. Cheng and Friedman
classified reputation systems into two categories,
\begin{enumerate}
\item symmetric reputation systems where the reputation score only depends on
  the network topology, popular reputation mechanisms such as
  PageRank\cite{page1999pagerank} and EigenTrust\cite{kamvar2003eigentrust} are
  examples of symmetric reputation systems, and
    \item asymmetric reputation systems where there some nodes are trusted and
      reputation scores are propogated through the trusted nodes, most OSN are
      examples of asymmetric reputation systems.
\end{enumerate}
The authors formally proved that symmetric reputation systems are vulnerable to
the Sybil attack. But in the asymmetric case, it is possible to construct a
Sybil-proof reputation system.

\subsection{The Sybil Attack in MANETs}
% TODO should we include this?

\subsection{The Sybil Attack in Reputation Systems}

\subsubsection{Reputation Systems}
\subsubsection{Attacks}

\subsection{The Sybil Attack in Online Social Networks}
\subsubsection{Online Social Networks}
\subsubsection{Attacks}

\subsection{TODO}
a test bed for sybil attacks\cite{irissappane2012towards}

Quantifying Sybil attack\cite{margolin2008quantifying}

%%% Local Variables:
%%% mode: latex
%%% TeX-master: "main"
%%% End:
