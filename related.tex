Many surveys on the sybil attack exist in the literature. We attribute much of
the initial findings to these surveys. In contrast to the existing work,
we also try to cover the possible attacks that can be mounted using sybils as
well as a wider range of defence mechanisms.

To the best of our knowledge, Levine et el. published the first survey of the
defence mechanisms\cite{marti2006taxonomy} in 2006. They found that the most
popular defence mechanism (in terms of the number of published work) at that
time is to use a certificate authority. The surveyed defence mechanisms
approximately cover sections \ref{sec:cert-authority},
\ref{sec:resource-testing} and \ref{sec:registration-fee} in this work.

Many more surveys were published after the introduction of OSN based sybil
defences beginning with SybilGuard. Mohaisen and Kim surveyed certificate
authority, resource testing and OSN (random walk) based
approaches\cite{mohaisen2013sybil}. They also compare the assumptions,
performance and many other properties. Rakesh et el. made a similar
survey\cite{rakesh2014survey}, they discuss six types of attacks in addition to
describing the defence mechanisms. Our work can be seen as an extended work of
these surveys---in terms of the possible attacks, and the defence mechanisms.

Koll et el. surveyed 8 defence mechanisms for OSN and analysed them in much more
depth\cite{koll2014state}. The authors experimentally show that increasing the
number of attack edges indeed makes the defence mechanisms less effective. More
interestingly, sybil tolerance systems\footnote{Sybil tolerance systems are
  those that are designed to limit the effectiveness of the sybils rather than
  detecting them.} such as Ostra (discussed in \autoref{sec:content-based}) and
SumUp (discussed in \autoref{sec:network-flow}) can still be effective when the
number of attack edges increase. The authors advise future defence mechanism
designers to use information in addition to simply the graph structure to detect
or tolerate sybils. In comparison, our work do not go into the same level of
depth, but provide a much broader spectrum of defence mechanisms.

Surveys of reputation often cover sybil attacks too \cite{marti2006taxonomy,
  hoffman2009survey, koutrouli2012taxonomy, selvaraj2012survey}. Although they
do not cover sybil attack in depth, these surveys provide a lot of insight from
a different perspective, especially for the possible attacks in reputation
systems.

%%% Local Variables:
%%% mode: latex
%%% TeX-master: "main"
%%% End:
