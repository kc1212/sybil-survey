We begin our survey by showing some alarming sybil attacks happening in the
real-world. First, we summarise a few amusing results from external studies,
specifically on the convenience and the scale of the sybil attack. Next, we
describe our experimental study on fake Twitter accounts.

\begin{figure}
  \centering
  \includegraphics[width=\linewidth]{boostlikes}
  \caption{Screenshot of the Facebook likes service page of boostlikes.com.}
  \label{fig:boostlikes}
\end{figure}

\begin{figure*}
  \centering
  \includegraphics[width=\textwidth]{socialformulae}
  \caption{Screenshot of the main banner on socialformulae.com.}
  \label{fig:socialformulae}
\end{figure*}

\subsection{External Studies}
\label{sec:external-studies}

In the introduction we showed how Twitter is used to manipulate public opinion
in elections. But this capability is not only accessible to campaigners with a
large budget. There are marketplaces where anybody can purchase false reputation
scores such as Twitter followers. BoostLikes shown in~\autoref{fig:boostlikes}
is a professionally presented website, it offers a large range of services
including Facebook likes, Twitter followers, Instagram followers and YouTube
views. SocialFormulae (\autoref{fig:socialformulae}) is a similar service but at
a much lower price point, one thousand Twitter followers is only \$9.99. 
% one thousand likes cost \$71 at the time of writing. 

SadBotTrue and its related website Socialpuncher publishes studies on social
media fraud. Two of their studies are particularly useful for demonstrating the
scale of the sybil attack and the obliviousness of Twitter. Firstly, there exist
a sybil group that consist of 3 million accounts. Since their creation, they
generated 2.6 billion tweets. Surprisingly, all the 3 million accounts were
created on the same day---22/10/2013, and the account names are simply numbered
sequentially~\cite{sadbottrue}. Such an obvious activity should be easily
detectable by Twitter, but these accounts are still not closed at the time of
writing. Secondly, the top--100 Twitter users have 523 million unique followers
between them, but 310 million are sybils, that is almost
60\%~\cite{socialpuncher}. Suppose the sybils all belong to the same attacker,
then they can effectively suppress the opinions of the real users.

Clearly, the defence mechanisms employed by popular social network websites are
not adequate to combat the sybil attack. If the sybils infiltrate even more of
our cyberspace, then it may become a form of censorship. Effectively taking away
our right to freedom of speech.

Speaking of censorship, around a million~\cite{tormetric} people use Tor (The
Onion Router)~\cite{dingledine2004tor} to access the uncensored internet when
living in authoritarian regimes such as China, or wish to uphold their privacy
from illegal mass surveillance by intelligence agencies. Unfortunately, Tor
suffered a sybil attack. In January 2014, 115 bogus relays joined the Tor
network. Six months later, it was discovered that those relays were using a
protocol vulnerability to deanonymise users and find the location of hidden
services. It is unclear to the Tor developers which users are affected or what
information was retrieved, thus it is assumed that users who used Tor between
that period are all affected~\cite{torsybil}. In fact, Tor depends on the fact
that majority of the relays are good to guarantee anonymity with a high
probability. If the network contains a large proportion of sybils, then users
can be easily deanonymised.

\subsection{Experiment}
In this section, we report our experience with buying fake Twitter accounts and
also the properties of those accounts. Twitter is selected for two reasons, (1)
it is one of the most popular social networks, (2) it is vulnerabile to the
sybil attack as described in \autoref{sec:external-studies}.

% We crawled Twitter to gain first-hand knowledge on the real-world sybils.
% Twitter was selected because it has an easy-to-use API and it is one of the
% major targets of sybils as mentioned at the top of this section.

\subsubsection{Buying Experience}
Before we can purchase sybils, we needed our own accounts. One was created on
the 25th of November 2016. We made sure our account has 0 tweets and is not
advertised in any other medium over the duration of this experiment. This
guarantees that all of its followers are sybils.

We purchased the ``1,000 Followers'' product from
CoinCrack\footnote{\texttt{https://coincrack.com/}} for \$9. No information was
asked except for an email address for them to deliver the receipt. Followers
started coming in almost immediately after we made our purchase. This is
especially interesting because we made the purchase using Bitcoin, but CoinCrack
did not wait for 10 minutes for our transaction to be confirmed. No more than 1
hour later, our brand new account with 0 tweets have accumulated 1,300
followers. These followers are certainly sybils because on CoinCrack's website
they state ``We broker followers in real time from dozens of engineers across
the globe who specilize in creating/maintaining large amounts of Twitter
profiles''.

\subsubsection{Sybil Properties}
Sybils themselves often form relationships so that they look like real users. To
find these relationships, we crawled our followers recursively for 72 hours
using
\verb!tweetf0rm!\footnote{\texttt{https://github.com/bianjiang/tweetf0rm}}. We
obtained 3 million nodes and 6 million edges at the end of the crawl. Every node
represents an account and every edge represents a ``following'' relationship. It
is certainly untrue that all 3 million nodes are sybils. But real users are
unlikely to follow fake accounts that do not create interesting or original
content, so for this work we assume the majority of them are sybils.

\autoref{fig:twitter-dd} shows the degree distribution graph. The majority of
the nodes, between $10^4$ and $10^6$ on the $y$-axis, have less than 10
followers. But a few accounts have a large number of followers, some of them are
in the order of 100,000. This raises the following question. Are these
``popular'' accounts truely sybils? To answer this, we inspect a few of these
accounts manually. What we discovered was that these ``popular'' accounts have
no meaningful or original tweets. Furthermore, they are created very recently,
some of them are as recent as November 2016. For instance,
\verb!@gf1av!\footnote{\texttt{https://twitter.com/gf1av}} joined Twitter in
November 2016 and has over 174K followers at the time of writing. Its tweets are
either retweets or meaningless sentences such as ``I love twitter so much''. In
fact, it is one of the followers of our account. These are evidences that
``popular'' accounts can also be sybils.

Now we switch our attention to the shape of the graph. It is a log-log graph,
where the ``straight'' line is the best fit line using the function $p(x) =
ax^k$, which is a power-law function. The degree distribution certainly does not
follow $p(x)$ closely. But to a first approximation, they do share some similar
properties, for example the decreasing trend. Note that the points with few
followers have little to no effect on $a$ or $k$ of $p(x)$, because the squared
residuals are low\footnote{The curve fitting is done in gnuplot, it uses the
  least-squares method, which uses squared residuals.}. Having a power-law
degree distribution is an indication that it is a real-world network. Since many
other networks such as the World-Wide Web, biological networks and other social
networks also have the same property~\cite{girvan2002community}. We do not have
a verifiable explination for this property, but we speculate that it is because
the attackers want to avoid detection, so they structure their sybil group in a
way that look like a group of real users.

% transitivity = 1
% we only show a small portion of all the sybils

\begin{figure}
  \centering
  \includegraphics[width=\linewidth]{twitter_dd}
  \caption{Graph of degree distribution for a graph with 3,248,093 nodes and
    6,062,427 edges. The average degree is 3.7. The number of nodes greater than
    the average is 120,075. The ``straight'' line is a best fit line, computed
    using gnuplot.}
  \label{fig:twitter-dd}
\end{figure}

\subsection{Summary}
We hope these examples demonstrate a big problem with the popular social network
websites and anonymous communication tool we use today, where a lot of sybils
controlled by an attacker can censor content and track user behaviour. In
\autoref{sec:attacks}, we zoom in on the practical attacks and explain them in
further detail.

%%% Local Variables:
%%% mode: latex
%%% TeX-master: "main"
%%% End:
