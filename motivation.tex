We begin our survey by showing some alarming sybil attacks happening in the
real-world. Social network and micro-blogging websites are popular platforms for
organisations to improve public relations and their reputation, but they are
also platforms to spread propaganda. 

\begin{figure}
  \centering
  \includegraphics[width=\linewidth]{boostlikes}
  \caption{Screenshot of the Facebook likes service page of boostlikes.com.}
  \label{fig:boostlikes}
\end{figure}

\begin{figure*}
  \centering
  \includegraphics[width=\textwidth]{socialformulae}
  \caption{Screenshot of the main banner on socialformulae.com.}
  \label{fig:socialformulae}
\end{figure*}

In the introduction we showed how Twitter is used to manipulate public opinion
in elections. But this capability is not only accessible to campaigners with a
large budget. There are marketplaces where anybody can purchase false reputation
scores such as Twitter followers. BoostLikes shown in~\autoref{fig:boostlikes}
is a professionally presented website, it offers a large range of services
including Facebook likes, Twitter followers, Instagram followers and YouTube
views. SocialFormulae (\autoref{fig:socialformulae}) is a similar service but at
a much lower price point, one thousand Twitter followers is only \$9.99. There
can be little doubt that those companies use automated bots to provide their
services.
% one thousand likes cost \$71 at the time of writing. 

SadBotTrue and its related website Socialpuncher publishes studies on social
media fraud. Two of their studies are particularly useful for demonstrating the
scale of the sybil attack and the obliviousness of Twitter. Firstly, there exist
a botnet that consist of 3 million accounts. Since their creation, they
generated 2.6 billion tweets. Surprisingly, all the 3 million accounts were
created on the same day---22/10/2013, and the account names are simply numbered
sequentially~\cite{sadbottrue}. Such an obvious activity should be easily
detectable by Twitter, but these accounts are still not closed at the time of
writing. Secondly, the top-100 Twitter users have 523 million unique followers
between them, but 310 million are bots, that is almost
60\%~\cite{socialpuncher}. Suppose the bots all belong to the same attacker,
then they can effectively suppress the opinions of the real users.

Clearly, the defence mechanisms employed by popular social network websites are
not adequate to combat the sybil attack. If the sybils infiltrate even more of
our cyberspace, then it may become a form of censorship. Effectively taking away
our right to freedom of speech.

Speaking of censorship, around a million~\cite{tormetric} people use Tor (The
Onion Router)~\cite{dingledine2004tor} to access the uncensored internet when
living in authoritarian regimes such as China, or uphold their privacy from
illegal mass surveillance by intelligence agencies. Unfortunately, Tor suffered
a sybil attack. In January 2014, 115 bogus relays joined the Tor network. Six months
later, it was discovered that those relays were using a protocol vulnerability
to deanonymise users and find the location of hidden services. It is unclear to
the Tor developers which users are affected or what information was retrieved,
thus it is assumed that users who used Tor between that period are all
affected~\cite{torsybil}. In fact, Tor depends on the fact that majority of the
relays are good to guarantee anonymity with a high probability. If the network
is infiltrated by numerous of sybils then users can be easily
deanonymised.

We hope these examples demonstrate a big problem with the popular social network
websites and anonymous communication tool we use today, where a lot of sybils
controlled by an attacker can censor content and track user behaviour. In
\autoref{sec:attacks}, we zoom in on the practical attacks and explain them in
further detail.

\subsection{Experiment}
We crawled Twitter to gain first-hand knowledge on the real-world sybils.
Twitter was selected because it has an easy-to-use API and it is one of the
major targets of sybils as mentioned at the top of this section.

\subsubsection{Setup}
A Twitter account was created on 25th of November 2016. We purchased the ``1,000
Followers'' product from CoinCrack\footnote{\texttt{https://coincrack.com/}} for
\$9. We made sure our account has 0 tweets and is not advertised in any other
medium, this guarantees that all of its followers are sybils. Followers started
coming in almost immediately after we made our purchase. No more than 1 hour
later, our brand new account with 0 tweets have accumulated 1,300 followers. To
find their relationships, we crawled the followers of sybils for 72 hours using
\verb!tweetf0rm!\footnote{\texttt{https://github.com/bianjiang/tweetf0rm}}
recursively for a depth of three---more than enough for our analysis below.
% We did not crawl 

\subsubsection{Results and Analysis}
We obtained 3 million nodes and 6 million edges by the end of crawling period.
We only consider the known sybils (the followers of our account) for two reasons.
(1) There is no guarantee how many of those 3 million nodes are actually sybils.
(2) Graph visualisation software cannot handle this volume of data under a
reasonable amount of time and computational resources.
% (3) hair-ball
\autoref{fig:twitter-graph} shows the result visualised using
Gephi~\cite{bastian2009gephi}. In particular, we use the OpenOrd layout
algorithm~\cite{martin2011openord} to capture the overall structure of the
graph. It structures the layout using hierarchical clustering so that we can
clearly see the communities. We observe the following properties.
\begin{enumerate}
  \item Many sybils are connected with each other, and many of them form a large
    community. This is most likely because the sybils need to be made to look like
    humans so that they can avoid detection. An account that follows thousands of
    users but has 0 followers would look suspicious.
  \item There exist 4 ``super sybils'' (large purple nodes at the centre of the
    graph) and they each have over 1000 followers, so they are connected to
    almost all the other sybils. One of them is
    \verb!@gf1av!\footnote{\texttt{https://twitter.com/gf1av}}, it also joined
    in November 2016 and has over 174K followers at the time of writing. We do
    not know the exact reason for this property, it may be another strategy to
    avoid detection.
  \item The ``1,000 Followers'' service is the cheapest Twitter service, so the
    attacker has many more sybils than what is pictured. Hence, the smaller
    communities may be in fact parts of a larger community.
\end{enumerate}

\begin{figure}
  \centering
  \includegraphics[width=\linewidth]{twitter_graph}
  \caption{Visualisation of the relationships of the sybils. Dark/light nodes
    are nodes with a high/low degree.}
  \label{fig:twitter-graph}
\end{figure}

From the results we see the sybils do not have the same characteristics, and
more importantly they form communities within themselves. In
\autoref{sec:defences}, we describe how to leverage the community property to
detect sybils or create sybil-resistant applications.

%%% Local Variables:
%%% mode: latex
%%% TeX-master: "main"
%%% End:
