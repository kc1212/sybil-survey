We begin our survey by showing some alarming Sybil attacks happening in the
real-world. Social network and micro-blogging websites are popular platforms for
organisations to improve public relations and their reputation, but they are
also platforms to spread propoganda. A recent article in the Atlantic described
how Twitter bots (Sybils) are shaping the 2016 US presidential
election\cite{atlantictwitterbots}. Over a third of pro-Trump tweets and almost
a fifth of pro-Clinton tweets, totalling at about 1 million, came from bots. The
article questions whether the bots are a threat to democracy because opinions of
real users are eclipsed by spam of bots.

\begin{figure}
  \centering
  \includegraphics[width=\linewidth]{boostlikes}
  \caption{Screenshot of the Facebook likes service page of boostlikes.com.}
  \label{fig:boostlikes}
\end{figure}

\begin{figure*}
  \centering
  \includegraphics[width=\textwidth]{socialformulae}
  \caption{Screenshot of the main banner on socialformulae.com.}
  \label{fig:socialformulae}
\end{figure*}

Using Sybils to manipulate public opinion is not only accessible to campaigners
with a large budget. There are marketplaces where anybody can purchase
reputation scores such as Twitter followers. BoostLikes shown on
\autoref{fig:boostlikes} is a professionally presented website, it offers a
large range of services including Facebook likes, Twitter followers, Instagram
followers and YouTube views\footnote{Facebook is possibly the largest social
  network website at the time of writing. Instagram is a social network website
  designed for sharing photos. YouTube is a video sharing website.}.
SocialFormulae (\autoref{fig:socialformulae}) is a similar service but at a much
lower price point, one thousand Twitter followers is only \$9.99. There can be
little doubt that those companies use automated bots to provide their services.
% one thousand likes cost \$71 at the time of writing. 

SadBotTrue and its related website Socialpuncher publishes studies on social
media fraud. Two of their studies is particularly useful for demonstrating the
scale of the Sybil attack on Twitter. Firstly, there exist a botnet that consist
of 3 million accounts. Since their creation, they generated 2.6 billion tweets.
Surprisingly, all of the 3 million accounts were created on the same day and the
account names are simply numbered sequentially\cite{sadbottrue}. Such an obvious
activity should be easily detectable by Twitter, but these accounts are still
not closed at the time of writing. Secondly, the top-100 Twitter users have 523
million unique followers between them, but 310 million are bots, that is almost
60\%\cite{socialpuncher}. Suppose the bots all belong to the same attacker, then
they can effectively suppress the opinions of the real users.

Clearly, the defence mechanisms employed by social network and micro-blogging
websites are not adaquate to combat the Sybil attack. If the Sybils infiltrate
even more of our cyberspace, then it may become a form of censorship.
Effectively taking away our right to freedom of speech.

Speaking of censorship, many users use Tor (The Onion
Router)~\cite{dingledine2004tor} to access the uncensored internet when living
in authoritarian regimes such as China, or intelligence agencies from doing
illegal mass surveillence. Unfortunately, Tor suffered a Sybil attack. In
January 2014, 115 relays joined the Tor network. Six months later, it was
discovered that those relays were using a protocol vulnerability to deanonymise
users and find the location of hidden services. It is unclear to the Tor
developers which users are affected or what information was retrieved, thus it
is assumed that users who used Tor between that period are all
affected\cite{torsybil}. In fact, Tor depends on the fact that majority of the
relays are good to guarantee anonymity with a high probability. If the network
is infiltrated by a large number of Sybils then users can be easily
deanonymised.

These example demonstrate a big problem with in the popular social network
websites and anonymous communication tool we use today. A lot of Sybils
controlled by an attacker can censor content and track user behaviour. In the
next section, we zoom in on the practical attacks and grouped them by the
underlying application.

%%% Local Variables:
%%% mode: latex
%%% TeX-master: "main"
%%% End:
